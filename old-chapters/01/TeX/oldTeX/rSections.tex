\subsection{Introductory methods in R}
\label{introductoryMethodsInR}

[[ Maybe this should go in a separate book or guide? Even if not, the next two paragraphs should go in an introduction for the book. ]]

R is a powerful statistical software that has been growing fast in both academia and industry. While the layout is very different from many program, this book will serve as a guide to getting started on each topic. If R is already downloaded and installed, then you are set to go. Otherwise see Appendix~ZZQ for instructions for obtaining and installing R online for free. Appendix~ZZQ also provides information on how to open R and how to (very easily) save all the data sets from this book to your computer.

By the way, R is free forever, just like this book.

When R opens, it will have a \term{prompt}: \rcom{>}. This means R is set to go and waiting for you to type something. To obtain the data, first open this book's package by typing the following command: \\

\rCom{library(osBook)}
\rCom{data(possum)}

(Hit ``return'' after each line. Also, don't type the greater-than sign (\rcom{>}), which here represents the prompt. If you get an error and you have typed everything correctly, see the footnote\footnote{Type \rcom{install.packages('osBook')} and pick a location near you. R will download \data{osBook} from the internet and get it ready to use. Once this is complete, retype \rcom{library(osBook)}.}. If you get a warning, ignore it.) The first command loads all of this book's data sets into R, making them easy for you to call up. The second command calls up the \data{possum} data set.

If you would like to examine the first observation of \data{possum}, type \\

\rCom{possum[1,]}

You will find one more variable has been added: \var{site}, which was described in Exercise~\exer{}. Verify the first row matches that of Table~\ref{possumDF} for the other variables. If you would like to examine another observation, put the observation number in place of \rcom{1}.
\begin{center}
\vspace{1cm}

[[ Provide some screenshot of R working. ]]

\vspace{1cm}
\end{center}


Maybe next you would like to see the entire data set. In this case, type \\

\rCom{possum}

Or perhaps you would just like to look at where each possum came from only: \\

\rCom{possum\$pop}

To examine any other variables of \data{possum}, replace \var{pop} above with a different variable name in the \data{possum} data set, e.g. \rcom{possum\$sex}. Two variables can be examined at a time by using a slightly modified notation: \\

\rCom{possum[,c('pop','sex')]}

A contingency table of this data can be constructed with relative ease in R using the \rcom{table} command around the previous result: \\

\rCom{table( possum[,c('pop', 'sex')] )}

A one-variable contingency table can be made via either \rcom{table(possum\$pop)} or \rcom{table(possum[,'pop'])}.

Here ends the introduction to examining data in R! In the next section, a basic introduction to graphics with R will be given. Try completing the following exercises by modifying the code above. \\

\begin{exercise}
Examine the data set \data{cars}. \\
(a) First look at the $13^{th}$ observed car. How many passengers can it hold? Answer in the footnote\footnote{Typing \rcom{cars[13, ]} gives only the information for the $13^{th}$ car, making it easier to identify the number of passengers as \resp{6}.}. \\
(b) What is the weight of the $25^{th}$ car? \\
(c) What is the miles per gallon for the $42^{nd}$ car?
%If you did not use the command \rcom{cars[42,]} for part (c), do so, and use a similar command for parts (a) and (b) as well if you have not already done so.
\end{exercise}

\begin{exercise}
(a) Type the command that will give you only the prices of all of the cars (recall the variable name is \var{price}). Answer in the footnote\footnote{\rcom{cars\$price}}. \\
(b) Type the command that will give you only the number of passengers for all of the cars (variable name: \var{passengers}). \\
(c) Create a one-variable contingency table of for the variable \var{driveTrain}. \\
(d) Recreate Table~\ref{typeDriveTrainTable} using the variables \var{type} and \var{driveTrain}. (Don't worry if the levels of the variables have the same order.)
\end{exercise}

