
\section{Introducing categorical data in R}
\label{categoricalDataInR}

If you have closed R, reopen the \rcom{openIntro} library and reload both the \data{possum} and \data{cars} data sets: \\

\rCom{library(openIntro)}
\rCom{data(possum)}
\rCom{data(cars)}

\begin{exercise} \label{getPopAndSexVariablesFromR}
Obtain the two-column data matrix (data frame in R) of just the variables \var{pop} and \var{sex} in the \data{possum} data set. The command is in the footnote\footnote{possum[ ,c('pop','sex')]}.
\end{exercise}

\subsection{Contingency tables in R}

A contingency table of the two variables from Exercise~\exer{getPopAndSexVariablesFromR} can be constructed with relative ease. Just apply the \rcom{table} command to the \var{pop} and \var{sex} data: \\
%A contingency table of these data can be constructed with relative ease in R using the \rcom{table} command around the previous result: \\

\rCom{table( possum[,c('pop', 'sex')] )}

A one-variable contingency table can be made via either \rcom{table(possum\$pop)} or \rcom{table(possum[,'pop'])}.

R output can be \term{stored} in a new variable that is created in R. For instance, create the variable \var{popSexTab} for the two-variable table above, and then access the table using the following commands: \\

\rCom{popSexTab = table( possum[,c('pop', 'sex')] )}
\rCom{popSexTab}

In this way, you can create your own subset of the data and easily save it to work with later. For instance, this variable can now itself become the input for another function, called \rcom{prop.table}:

\rSSSpaceTop

\includegraphics[height=70mm]{ch1/rScreenshots/usingProp-Table.png}
\rSSSpaceBottom

The last two commands do the same thing, even though \rcom{margin=} was left out. Why is this? \\

\begin{tipBox}{\tipBoxTitle{Leave out argument names in functions}
If the arguments to an R function are given in their default order (which is not always obvious), then R does not need the argument names and only the argument values.}
\end{tipBox}

\begin{exercise}
Does \rcom{margin=1} give row or column proportions? Try using \rcom{margin=2} (or just \rcom{2} as the second argument).
\end{exercise}

\subsection{Creating bar and mosaic plots in R}

To make a mosaic plot, feed the function \rcom{mosaicplot} a table: \\

\rCom{mosaicplot(popSexTab)}.

\begin{exercise}
(a) Make a table of the \var{site} and \var{sex} variables in R, (b) assign this table to a variable (give it a name), and (c) plot the table in a mosaic plot. One solution is in the footnote\footnote{(a \& b) \rcom{siteSexTab = table( possum[ ,c('site','sex')] )} (c) \rcom{mosaicplot(siteSexTab)}}.
\end{exercise}

A bar plot can be made from a table of a single variable: \\

\rCom{barplot(possum\$sex)}

\begin{exercise}
Make a bar plot of the \var{driveTrain} variable from \data{cars}.
\end{exercise}

\subsection{Review problems for R}

\begin{exercise}
Complete the following three exercises by calling only a single row from the \data{cars} data set. \\
(a) How many passengers can the $13^{th}$ car hold in the \data{cars} data set? \\
(b) What is the weight of the $25^{th}$ car? \\
(c) How many miles per gallon does the $42^{nd}$ car get?
\end{exercise}

\begin{exercise}
Make a scatterplot of the variables \var{mpgCity} and \var{weight} for the \data{cars} data set. Make the plot so that \var{weight} is along the horizontal axis.
\end{exercise}

\begin{exercise}
Make a dot plot, histogram, and box plot of \var{totalL} from the \data{possum} data set. Are there any suspected outliers?
\end{exercise}
