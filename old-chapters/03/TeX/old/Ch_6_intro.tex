
\chapter{Inference for categorical data}
\label{inferenceForCategoricalData}

Chapter~\ref{inferenceForCategoricalData} introduces inference in the setting of categorical data, which is the realm of proportions and contingency tables. We use these methods to questions like the following:
\begin{itemize}
\item What fraction of the American public trusts the federal government most of the time?
\item What is the difference in support for Congressional action on healthcare when comparing Democrats and Republicans?
\item Is their racial bias in who is selected to serve on juries in a local community?
\end{itemize}
These and other questions motivate the use of proportions and contingency tables, topics we will cover this chapter.

We will find that the inference methods we learned in previous chapters will be very useful in this new setting. For example, sample proportions are well characterized by a nearly normal distribution when certain conditions are satisfied; in such cases, we apply our usual confidence interval and hypothesis testing tools with the normal model. In other instances, such as those with contingency tables or when the sample size conditions are not met, we will use a different distribution, though the general ideas remain the same.




Categorical data can be summarized by binning the data into groups, where each group makes up some proportion of the whole. In the simplest case, we seek to estimate a single proportion using a confidence interval or perform a hypothesis test for som.

There are three primary settings that we discuss:
\begin{description}
\item[Single proportion.] We seek to estimate a single proportion $p$, which we estimate using a sample proportion. When the observations are independent and the sample size is sufficiently large, the sample proportion follows a nearly normal distribution.


, or a difference in proportions $p_1 - p_2$, where the observations are independent and sample sizes are large. In this setting, the sample estimates of 
\item[Difference in proportions.] 
\item[Contingency table.] 
\end{description}


 In the simplest setting, we may seek to understand the proportion in a single group: $p$. We can estimate $p$ using a sample proportion, which is labeled $\hat{p}$. We will find that the sampling distribution of a sample proportion, just like a sample mean, tends to be nearly normal under certain conditions, and this is the subject of Section~\ref{singleProportion}. Similarly, a difference in sample proportions $\hat{p}_1 - \hat{p}_2$ also tends to be nearly normal when each proportion is nearly normal and the two samples are independent, and we discuss this setting in Section~\ref{differenceOfTwoProportions}.



It is also common to examine the difference in proportions: $p_1 - p_2$. 

Such a situation is the subject of Section~\ref{}.



Categorical data can be summarized by binning the data into groups, where each group makes up some proportion of the whole. For example, proportions naturally arise in political polling when people are asked which candidate they support. In the simplest setting, we 

For instance, we may wonder, what proportion of the American public trusts the federal government? Here groups could be  To answer this and similar questions, we collect a sample. The sample proportion provides a point estimate of the true proportion, with some uncertainty.

Like the sample mean, the sample proportion can be well-modeled using the normal distribution under certain conditions. Many of the details in Sections~\ref{} and~\ref{} will look very familiar to the setting of sample means.

Sections~\ref{} and~\ref{} will provide

e can also model the difference of such proportions using 

A sample of categorical data may be summarized by binning data into groups, where each group makes up some proportion of the whole. We estimate these proportions using data, usually by taking a simple random sample. Point estimates for proportions, like point estimates for means, vary from one sample to the next. Our task is to characterize and quantify how this sampling variability affects the accuracy of the point estimates with the goal of making inference about a population or process.

We will find that the inference methods we utilize for means also work for categorical data. In fact, sample proportions can even be modeled with the normal distribution when the sample size is sufficiently large. When comparing many proportions simultaneously,

The ideas will remain the same, while the details change. proportions and differences in proportions, when the sample size is sufficiently large.

sampling distribution of a sample proportion, like a sample mean, tends to be nearly normal when the sample size is sufficiently large. We may also like to compare many proportions simultaneously, which we 

