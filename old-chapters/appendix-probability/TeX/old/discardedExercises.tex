%\begin{exercise}
%There are 52 unique cards in a deck and you think specifically about one of them (for example, you think about \resp{4$\diamondsuit$}). If one card is selected at random, then what is the probability of selecting your card? Answer in the footnote\footnote{Each of the 52 cards has an equal chance of being picked. Thus, the probability the selected your card is $1/52$.}.
%\end{exercise}

%\begin{exercise}\label{cardSuit}
%The cards in the deck are split into four separate \term{suits} (club: $\clubsuit$, diamond: $\diamondsuit$, heart: $\heartsuit$, spade: $\spadesuit$) with 13 cards per suit. One card is selected at random. (a) What is the probability the card is a diamond? (b) What is the probability the card is a heart? (c) Let $A_1$ represent the event the card is a diamond and $B_1$ represent the event the card is a heart. Are the events $A_1$ and $B_1$ disjoint? (d) Because $A_1$ and $B_1$ are disjoint, use the Addition Rule and parts (a) and (b) to determine the probability the card is a diamond or a heart.
%\end{exercise}

%\begin{exercise}
%Each suit has its 13 cards numbered \resp{2}, \resp{3}, ...,\resp{9}, \resp{10}, \resp{J} (jack), \resp{Q} (queen), \resp{K} (king), \resp{A} (ace), i.e. there are four cards of each number with each being a different suit. If one card is selected randomly from the deck, what is the probability it is one of the following: \resp{4$\spadesuit$}, \resp{8$\diamondsuit$}, or \resp{J$\diamondsuit$}? Answer in the footnote\footnote{These three outcomes are disjoint and each has probability $1/52$. Therefore the probability of selecting a card in the set $\{$\resp{4$\spadesuit$}, \resp{8$\diamondsuit$}, \resp{J$\diamondsuit$}$\}$ is $P($\resp{4$\spadesuit$} or \resp{8$\diamondsuit$} or \resp{J$\diamondsuit$}$) = P($\resp{4$\spadesuit$}$)+P($\resp{8$\diamondsuit$}$)+P($\resp{J$\diamondsuit$}$)=3/52$.}.
%\end{exercise}

%\begin{exercise}\label{facecard}
%Any card that is a \resp{J}, \resp{Q}, or \resp{K} is called a \term{face card}. If one card is randomly selected, what is the probability it is a face card?
%\end{exercise}

%\begin{exercise}\label{exerToComputeDisjointEventsFromTwoEarlierExercises}
%One card is selected at random. Let $A=\{$\resp{4$\spadesuit$}, \resp{8$\diamondsuit$}, \resp{J$\diamondsuit$}$\}$ and $B$ represent the set of all face cards. What is $P(A$ or $B)$? Answer in the footnote\footnote{The events $A$ and $B$ are disjoint so we can use the Addition Rule:
%\begin{eqnarray*}
%P(A\text{ or }B) = P(A) + P(B) = 3/52 + 12/52 = 15/52
%\end{eqnarray*}}.
%\end{exercise}

%\begin{exercise}
%How did you apply the Addition Rule in Exercises~\exer{cardSuit}-\exer{exerToComputeDisjointEventsFromTwoEarlierExercises}?
%\end{exercise}

%\begin{exercise}
%For each of the following probability distributions, determine what ``$x$'' should be so they are each probability distributions. (a) $(0.1, 0.3, x)$, (b) $(0.4, x, 0.2, 0.1, 0.1)$, and (c) $(0, x, 0.3, 0.1)$. Answer for (a) in the footnote\footnote{Must find $x$ such that $0.1+0.3+x=1$, i.e. $x=1-0.1-0.3=0.6$.}.
%\end{exercise}

%\begin{exercise}
%Are an event and its complement always disjoint? Answer in the footnote\footnote{Yes! If an outcome is represented in the event $A$, then this outcome will not be in $A^c$.}.
%\end{exercise}

%\begin{example}{Suppose you are on a game show and given three doors. Behind one random door is a car and behind the other two doors are goats. Suppose you pick door 1. Then the host, who knows where the car is, reveals that a goat is behind door 3. (No matter the door the contestant picks, the host always shows a hidden goat.) He then asks you, ``Do you want to stay with door 1 or pick door 2?'' Use Bayes' Theorem to show that you should pick door 2. Exercise citation in the footnote\footnote{This question was originally posed by Steve Selvin in a letter to \emph{The American Statistician} in August 1975, Vol. 29, No. 3.}.} \label{monteHallExercise}
%Either our initial guess was right -- since we had a one-in-three chance of being right, $P(\text{right}) = 1/3$ -- or we were not right $P(\text{not right}) = 2/3$.
%\end{example}

%\begin{figure}
%\begin{center}
%\includegraphics[height=35mm]{02/figures/monteHallWithDoor3Goat} \\
%\addvspace{2mm}
%\begin{minipage}{\textwidth}
%\caption[monteHallWithDoor3Goat]{There are three doors: one with a car behind it and two with goats behind them. You pick door 1, and the host reveals that door 3 was hiding a goat. Given the option, should you now pick door 1 or door 2?\vspace{-5mm}\footnote{Image from Wikipedia, originally created by user Cepheus and modified by user Father Goose.}}
%\label{monteHallWithDoor3Goat}
%\end{minipage}
%\end{center}
%% This image is under a Public Use license.
%\end{figure}

%\begin{table}[ht]
%\begin{center}
%\begin{tabular}{l c c r}
%\hline
%				& A or B on midterm & $\leq$C on midterm & Total \\
%\hline
%A or B on final 		& 11				& 9				 & 20 \\
%$\leq$C on final 	& 6				& 7				 & 13 \\
%\hline
%Total				& 17				& 16				 & 33 \\
%\hline
%\end{tabular}
%\end{center}
%\caption{Comparing midterm and final grades.}
%\label{midtermFinal}
%\end{table}

%\begin{exercise}
%Table~\ref{midtermFinal} represents student grades in a section, and 4 students are selected at random from this section without replacement. (a) What is the probability the first student earned an A or B on both the midterm and final? (b) What is the probability that all four students earned an A or B on both the midterm and final? (c) What is the probability the first three selected earned an A or B on the final and the last student earned a C or lower on the final? (d) Would your answers to (a)-(c) be the same if the students were sampled with replacement? Why or why not?
%\end{exercise}

